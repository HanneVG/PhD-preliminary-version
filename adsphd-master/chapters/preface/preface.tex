\chapter*{Acknowledgements}                                  \label{ch:acknowledgements}


Many similarities can be drawn between rivers and the trajectory leading to a PhD degree. Both start small and end up in something big such as the ocean or a PhD thesis. Comparable to a river which is formed and influenced along its journey, I have been influenced and assisted during the past four years by many people, which I (once more) would like to thank.
\bigskip 

First of all, I want to thank my promoters Steven, Gerard and Okke for being the rain over the river: without rain, the river would never have started and the rain keeps on feeding the river throughout the entire course. Steven and Gerard brought in their own experience, expertise and methods and this whole range from thunderstorms to drizzle and even drought, have guided and inspired me to become the researcher I am now. It was a privilege to work with you and I look forward to cooperate in the future, sooner or later. 
\bigskip

The supervisory committee and the jury members have been the infrastructure on the river. Just as dams stop the river for a while and give it time to rest, likewise the meetings with the supervisory committee have been a moment to reflect and to get guidance. Therefore, I'm grateful to Gert Verstraeten and Roel Merckx to make time for it. The feedback from the jury members can be compared with canalization works: traightening up the meandering thoughts in the manuscript to improve the overall quality of the work. I sincerely appreciate the feedback which improved the content of this thesis. Likewise, I appreciate the effort of the external jury members, Paolo Paron and Jack Middelburg, to undertake the journey to Belgium to be present at my preliminairy or public defence.
\bigskip

I would not be at this point of my PhD without the help of many others. I would like to compare them with the floodplains: they take up the water of the river when its capacity is reached and there can be a mutual exchange of material. Similarly, I could count on these people when I couldn't handle it all by myself or when I didn't have the data or the knowledge. Therefore, I want to thank the staff of Kenya Wildlife Service, especially at Garissa, who provided a immeasurable assistance while I was staying in Garissa and by performing the sampling when I was at Garsen. I am grateful for making me feel at home in their offices and I couldn't have done it without their help. I am also thankful to the staff of the Water Resources Management Authority at the Garissa branch, which provided discharge data, did measurements with the ADCP and assisted with the daily sediment sampling. Furthermore, I am grateful to the researchers of IRD, who set up the discharge measurements in Garsen and shared their data. In Belgium, I found some floodplains with the colleagues from the Institute for Nuclear and Radiation Physics who assisted me in my adventure with the Ge-detector. I am grateful for the assistance in the analysis of the total alkalinity by colleagues at the University of Li\`ege, and for the analysis of nutrients and elements by colleagues from the division of Soil and Water Management at the KU Leuven. Furthermore, I want to thank some individuals who didn't fit yet in one of the previous organizations: Peter Meulenijzer for the introduction to ADCP's, Paolo Paron for the information about meandering rates in the Tana River, Alberto Borges for the fruitful discussions and Filip Meysman for the introduction to reactive transfer modeling.
\bigskip 

The Tana River has also many tributaries, irrigation channels and animals such as hippos, crocodiles, fish, elephants, \ldots. These roles are assigned to all my colleagues in the geo-institute. As tributaries, you have improved my thesis with insights, ideas, comments or practical help. I hope that from my side, I have been an irrigation channel to all of you, by helping you in one way or the other. Finally, we also had a lot of fun together. That is how  you have been the animals in the river. Therefore, I would like to thank Zita, Fred, Tine, Trent, Christian, Fredrick, C\'edric, Daan, Eric, Lore,  Christophe, Valentijn, Jos, Helena, Hanne, Joachim, Dries, Nele, Mattias, Laura, Audrey, Jennifer, Kristine, Lameck, Nirman, Pablo, Reinout, Vicente, Karen, Eline, Stefaan, Jan, Jan, Guido, Seppe, Dirk, Shetie, Ayalew, Vladimiro, the colleagues from the division of geography, the staff at the reception and the ICT team. If somebody feels forgotten, I'm really sorry about that. Just know that I appreciated your input.
\bigskip

Besides the animals that provide some attraction to the river, there are also many boats. Those represent all the people who didn't have a lot to do with the research, but who provided relaxing distraction to me during the (sometimes tough) period of my PhD. Although it's very hard to name everybody, I want to say I appreciated the company of my housemates, the people in the Red Cross (both in Belgium and Kenya), the musicians in the orchestra, my fellows in the capoeira, all the police men and women in Kenya who were happy to talk to me and were caring for me, all the other men, women and children who I shared a nice time with in Kenya and all the other people who don't fit in any of the groups but who nevertheless have somehow made an impact on my motivation during my PhD.
\bigskip

And at last, but for sure not the least, I want to thank my family, my mother, father, sisters and brother. You have supported me when I was going to the --not really safe-- area along the Tana River. You have provided me with assistance, for example by fitting those 92 kg of fieldwork materials nicely in four boxes and bringing me to the airport. You have motivated me by listening to my thousand-and-more stories and worries. You have encouraged me when I was down. All of you, where the sunshine, the moon and the stars over the long river that was my PhD.
\bigskip

To all of you, I want to say: "THANK YOU, DANK U WEL, ASANTE SANA!"

Naomi


\vfill
This research was funded by the KULeuven Special Research Fund, the Research Foundation Flanders (FWO-Vlaanderen, project G024012N), and ERC-StG 240002
(AFRIVAL, http://ees.kuleuven.be/project/afrival/).

%%%%%%%%%%%%%%%%%%%%%%%%%%%%%%%%%%%%%%%%%%%%%%%%%%
% Keep the following \cleardoublepage at the end of this file, 
% otherwise \includeonly includes empty pages.
\cleardoublepage

% vim: tw=70 nocindent expandtab foldmethod=marker foldmarker={{{}{,}{}}}
