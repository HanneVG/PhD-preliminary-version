\chapter{Discussion, conclusion and outlook}\label{ch:conclusion}

As formulated in the introduction (\autoref{ch:introduction}) the goal of this research project was to develop and evaluate a parsimonious, widely applicable agro-hydrological model that can be used to evaluate the effect of agricultural management from field to catchment scale. This final chapter evaluates in how far the developed AquaCrop-Hydro model meets the targeted criteria, and which aspects require further model development. Finally, the findings on the applicability of this new agro-hydrological model for supporting sustainable land and water management decisions are discussed.

\section{Assessment of AquaCrop-Hydro}

\subsection*{Criterion 1: The model simulates crop production and water productivity at field scale, as well as hydrological processes and water availability at catchment scale}

\textit{Assessment: Being a combination of the AquaCrop crop water productivity model and a conceptual hydrological model per the generalized structure by \textcite{willems2014a}, AquaCrop-Hydro fulfils this first criterion to a large extent.}

AquaCrop-Hydro is indeed able to simulate crop water productivity as well as hydrological processes and water availability. The AquaCrop submodel deals with simulation of crop development and productivity as well as the soil water balance. The conceptual hydrological model, on the other hand, simulates overland flow, interflow and baseflow which form the river discharge at the catchment outlet. The latter is considered to be an estimate for water availability in the catchment. 

Due to the strong conceptual nature of both submodels, not all hydrological processes are described in a detailed physically based manner. The soil water content, and corresponding processes of infiltration, transport, and percolation of water in and out the soil profile are simulated in a physically based way. By contrast, generation of surface runoff is based on the curve number method, which is determined by empirical parameters. Also, overland flow, interflow and baseflow are simulated using empirical linear reservoir functions. Hence, AquaCrop-Hydro describes small scale processes that determine the volume of water in a physically based way, whereas the processes that determine the temporal variation of water flow at a larger scale are handled in a conceptual way. Due to this conceptualization, the model provides little information on the nature and functioning of the latter processes.

Furthermore, the detail with which some hydrological processes are described in AquaCrop-Hydro might not be sufficient for certain applications. First, it was demonstrated in \autoref{ch:aquacrophydro} that surface runoff generation was not described very accurately by AquaCrop-Hydro. Due to the model's daily time step, the impact of sub-daily rainfall intensity on surface runoff generation is neglected. This limits model application for flood modelling. However, if rainfall data with smaller time steps (e.g. 15 minutes) are available, surface runoff simulation might be improved by adjusting the model. This requires replacing the daily AquaCrop runoff calculation procedure by a procedure that is more suitable to small time steps (e.g Green and Ampt method). The surface runoff procedure is then no longer part of the AquaCrop submodel, but remains closely linked to it via the soil water balance. 

Second, also the interaction between the unsaturated and saturated zone is not well described in AquaCrop-Hydro. The current model structure, proposed in \autoref{ch:aquacrophydro}, simulates that a part of the water percolating out of the soil profile goes to the groundwater table and reaches the river via baseflow. In the opposite direction, it is not the baseflow reservoir that determines the capillary rise, but the user-specified depth of the groundwater table below the soil surface. The latter is a remnant of the AquaCrop soil water balance model, which considers the groundwater as a user-defined boundary condition. The potential discrepancy between the user-specified input of the groundwater table depth on the one hand and simulation of the groundwater volumes on the other hand makes that a closed water balance can not be assured. In addition, it is practically infeasible to obtain sufficient data to specify the time variable groundwater table depth for each land unit within the catchment, and even impossible for future time horizons.

This issue regarding the unsaturated-saturated zone interaction is no problem when the model is applied to areas where the groundwater table is deep so that the interaction between the unsaturated and saturated zone is one-directional. For example, in \autoref{ch:aquacrophydro} and \ref{ch:achtoep} the proposed model structure was applied for the Plankbeek catchment, where capillary rise from the groundwater was neglected. By contrast, for other areas where a shallow groundwater table significantly affects the soil water
balance, the proposed model structure may be less suitable. For those areas it is crucial to consider the unsaturated-saturated zone interaction simultaneously in both directions. 

Clearly, AquaCrop-Hydro's procedures for simulating the upward flow from the saturated to the unsaturated zone need to be updated to make the model applicable to groundwater-dominated catchments. First, the simulated volume of capillary rise needs to be subtracted from the baseflow recharge at each time step. Second, instead of a user-specified value, the depth of the groundwater table could be automatically derived from the simulated volume of water in the baseflow reservoir at each time step. To preserve model simplicity, an empirical approach seems most suitable to do so. The generalized conceptual model structure by \textcite{willems2014a}, which was partly implemented in AquaCrop-Hydro, does not account for the upward flow between the baseflow and soil reservoir. But, inspiration can be found in the NAM conceptual model \parencite{dhi2009}. NAM simulates capillary rise between a groundwater storage reservoir and root zone storage reservoir based on the simulated relative water content of the root zone storage reservoir and depth of the groundwater table, as well as a parameter corresponding to the depth of the groundwater table for a capillary flux of \SI{1}{mm/d}. It should be noted that by implementing such an automatic calculation of the groundwater table depth, model complexity and data requirements will increase. For that reason, the flexibility to discard this model component when applying AquaCrop-Hydro for catchments without a shallow groundwater table as well as the use of a constant groundwater table depth should remain an option.

Furthermore, AquaCrop-Hydro is indeed able to consider both field and catchment scale processes. Crop development and production are simulated on point-basis, representing one homogeneous field or land unit. River discharge and corresponding subflows are simulated at catchment scale using a lumped approach. The soil water balance is simulated at both scales. Scaling up the soil water balance from field to catchment scale is done as suggested by \textcite{wesseling2006}, using a semi-distributed approach. Because the catchment is not represented in a fully distributed manner, AquaCrop-Hydro neglects spatial connectivity that is important for some hydrological processes. Neither the interaction between the soil water balance of various land units, nor the spatial variation of some flow components is considered in AquaCrop-Hydro. For example, surface runoff generated in one field can not be considered as run-on in a neighbouring field. Also, there is no difference between the recession time of surface runoff generated in land units close to the catchment outlet as compared to surface runoff generated in the upstream area. By contrast, when studying the hydrological impact of urbanization, \textcite{poelmans2010} found that surface runoff calculations are very sensitive to accurate spatial information, especially for small catchments.

Switching to a fully distributed approach for simulation of the catchment soil water balance could improve the spatial representation of hydrological processes across scales. However, this would not only increase data requirements but is also technically difficult with the current AquaCrop software. Crop models, such as AquaCrop, operate at point-basis because they were initially developed to study crop production in a single homogeneous field. Since large scale scenario analysis of agricultural production is becoming more common, modelling frameworks have been developed or adjusted to deal with spatial heterogeneity. However, most of these frameworks run a set of parallel point-based simulations for different spatial units which do not interact which each other \parencite{holzworth2015}. Examples are the AquaCrop-GIS framework \parencite{lorite2013} or the GeoSim toolbox \parencite{thorp2013}, which enable AquaCrop simulations for multiple locations using geospatial data within a geographic information system. To date, only APSIM \parencite{keating2003} and CropSyst \parencite{stockle2003} allow simultaneous simulation of multiple points with dynamic interaction. Similar functionality would need to be added to the AquaCrop model in order to improve spatial representation of hydrological processes in AquaCrop-Hydro.

\subsection*{Criterion 2: The model considers the effect of management and environmental changes on crop transpiration and crop (water) productivity, as well as catchment hydrology} 

\textit{Assessment: Due to the physically based nature of the AquaCrop submodel, AquaCrop-Hydro fulfils this second criterion, although improvements are possible.}

As discussed in \autoref{ch:aquacrop} and demonstrated in \autoref{ch:fertility} to \ref{ch:aquacropscen}, AquaCrop enables simulation of the effect of agricultural management on crop canopy development, crop transpiration, crop biomass production, crop yield, crop water productivity and various components of the soil water balance in a cultivated field.

The agricultural management practices that can be considered include crop management, soil management, field surface management, mulches, soil fertility management and weed management practices. Unfortunately, not all of them are implemented in AquaCrop as a management practice, which makes their use intricate. A significant improvement in that matter was made in AquaCrop 5.0 for field surface management practices that affect surface runoff. Whereas in previous AquaCrop versions the curve number was solely a soil parameter, the effect of field surface management on this parameter is now explicitly implemented in the management module. Further improvements could be made by implementing runoff agriculture as a field management practice in AquaCrop. This avoids the parallel simulations that are currently required.

Furthermore, with the exception of weeds, neither the effect of pests (including animal pests, pathogens and viruses) nor diseases can be automatically simulated with AquaCrop. However, when field observations of the impact of such pests at a certain time in the growing season are available, the simulated canopy cover or biomass can be manually updated to account for potential yield losses in the simulation of the remaining growing season. Because of the complexity and huge variety of processes through which pests affect crop development and production, their automatic consideration in AquaCrop seems difficult without compromising the model's parsimonious nature. 

Next to agricultural management, also the agronomic impact of climate change can be simulated with AquaCrop as discussed in \autoref{ch:aquacrop} and demonstrated in \autoref{ch:achtoep}. The effect of climate change is considered by adaptation of the weather and \COtwocon input variables. These climate changes affect the simulated crop canopy development, crop transpiration, crop biomass production and crop (water) productivity according to the procedures that describe crop responses to abiotic factors such as water availability, air temperature and \COtwo concentration (\autoref{sec:ch2_abioticfac}).

It should be noted that these procedures show room for improvement. The scenario analysis in \autoref{ch:aquacropscen} indicated that the simulated crop responses to water stress did not always meet expectations. Also \textcite{castanedavera2015,ahmadi2015,abedinpour2012} have mentioned that AquaCrop performs less good under conditions of (extreme) water stress. From \autoref{ch:aquacropscen} it appears that water stress is not well represented because it is derived from the average soil water content over the whole root zone. This average soil water content does not take into account water and root distribution within the soil profile, which strongly defines the level of water stress experienced by a crop. It is, for example, highly unlikely that a crop would suffer from water stress when the topsoil with high root density is very wet even if the subsoil is dry. Clearly, AquaCrop calculation procedures for water stress should be improved by calculating water stress based on a weighed average soil water content, using the root distribution as a weighing factor. In addition, expansion of the root zone should only be slowed down by the presence of a restrictive soil layer and not completely stopped as it is currently the case.

Next, also the procedures to simulate crop responses to air temperature can be improved. \autoref{ch:weed} as well as \autoref{ch:achtoep} showed that crop development and production of winter wheat was not simulated realistically because AquaCrop does not consider processes such as cold acclimation, vernalization and dormancy. This issue was already mentioned by \textcite{vanuytrecht2014a}, but so far no new procedures for winter crops have been implemented. Furthermore, \autoref{ch:achtoep} indicated that the effect of heat stress on crop production is under-estimated, especially the impact of extreme temperatures. A correction of the harvest index to high temperatures, as proposed by \textcite{villalobos2015}, would be a first step to improve model simulations. 

Aside from altered daily weather conditions, climate change will also bring about increased incidence of extreme events such as droughts, heavy precipitation, heat waves, extreme cold, wind and hail storms \parencite{ipcc2014}. To date most crop models, including AquaCrop, can not take such extreme events into account. For that reason, the impact of climate change simulated by AquaCrop-Hydro, as presented in \autoref{ch:achtoep} for the Plankbeek catchment, should be interpreted with care. Future research, such as conducted in the framework of the MODEXTREME (MOdelling vegetation responses to EXTREMe Events) project \parencite{villalobos2015}, is extremely important to improve model structures and tackle this problem.

Although this research focussed on climate change, also other environmental changes can be simulated with AquaCrop-Hydro. For example, the effect of soil degradation or changes to the groundwater table can be considered by changing the relevant AquaCrop input parameters. Furthermore, also simulation of land uses changes can be done by adaptation of the relative area of each land unit within the catchment when scaling up the soil water balance from field to catchment scale.
 
Moreover, since AquaCrop automatically adjusts the simulated canopy cover and soil water balance to agricultural management and environmental changes, also the catchment soil water balance and water availability as simulated by AquaCrop-Hydro account for the effect of agricultural management and environmental changes. 

This dynamic approach of AquaCrop-Hydro is crucial for accurate simulation of the catchment soil water balance and water availability. It was quantified for the Plankbeek catchment in \autoref{ch:achtoep} that ignoring management adaptations in response to climate change could lead to underestimation of future water availability during summer months of up to 12\%. Likewise, it was quantified that neglecting crop responses to climate change, as often done in conceptual hydrological models, could lead to overestimation of the catchment's evapotranspiration by about 10\% during future summer months. Because evapotranspiration is a very large component of the soil water budget, such overestimation could cause considerable errors to the simulated discharge at the catchment outlet. This error could be further quantified by comparing the discharge simulated by the dynamic AquaCrop-Hydro model under future climatic conditions with results obtained by a static conceptual hydrological model. Conceptual models defined according to the VHM approach by \textcite{willems2014a} are most appropriate for that purpose given their large similarity with AquaCrop-Hydro.

Despite the dynamic approach of AquaCrop-Hydro, it should be noted that neither agricultural management nor environmental changes directly affect the simulated water routing at catchment scale. Also in \autoref{ch:achtoep}, the effect of climate change and related management adaptations was simulated by adjusting only the field scale input variables and parameters. The lumped hydrological model parameters, that define the recession times of the various subflows and the baseflow-interflow proportion in the catchment, were left unchanged. 

In reality, one would expect that the catchment response behaviour changes when management or environmental changes affect catchment characteristics. In particular, the overland flow recession time will be affected by changes to the soil surface characteristics. For example, field surface management such as ridges increase surface roughness and consequently slow down overland flow and thus increase the recession time. Also, land use changes or altered crop growth patterns due to climate change modify the soil surface and thus influence overland flow recession times. Furthermore, agricultural management could affect the soil's hydraulic conductivity and consequently the interflow recession time. However, this can already be accounted for by AquaCrop during simulation of drainage out of the soil profile, so that it would not require additional adjustment of the interflow recession constant of AquaCrop-Hydro. In addition, management practices might affect soil characteristics that determine the capacity to transport water via lateral drainage (interflow). For example, interflow might decrease if soil layers inhibiting water percolation are broken up during land preparation. As such, management can also affect the interflow-baseflow proportion. Finally, as opposed to overland flow and interflow, it is unlikely that baseflow recession time would be affected by agricultural management or environmental changes.

It is clear that abandoning the assumption of static catchment response behaviour to agricultural management could increase accuracy of model results, especially for overland flow and interflow. However, the empirical nature of AquaCrop-Hydro's hydrological model parameters make it difficult to adjust them according to the management practices of the individual landunits. Nevertheless, it might be possible to relate conceptual model parameters to observable catchment characteristics. An example was set by \textcite{tran2016} who disaggregated lumped catchment scale conceptual models to higher spatial resolutions based on physical catchment characteristics such as topography, land use and soil type. Such a link between empirical model parameters and observable catchment characteristics could support the adaptation of the conceptual model parameters to agricultural management.

\subsection*{Criterion 3: The model requires a feasible amount of easily obtainable input data and parameters to be calibrated, without compromising much the accuracy of the model results}
\textit{Assessment: Due to the parsimonious nature of both submodels, AquaCrop-Hydro fulfils this third criterion.}

Both submodels require relatively few input or calibration data. AquaCrop's low requirements for easily obtainable input data was extensively discussed in \autoref{ch:aquacrop} and demonstrated by simulating crop production for data-scarce experimental sites in Ethiopia, Bolivia and Nepal (\autoref{ch:fertility} and \ref{ch:weed}). Also, model comparison studies by \textcite{abisaab2015, castanedavera2015,todorovic2009} have confirmed that AquaCrop has lower data requirements than other crop models such as CropSyst, CERES and WOFOST. Next to the AquaCrop input data, only a series of daily river discharge observations is required to calibrate the hydrological model parameters. Such time series is commonly available, even in data-scarce regions.

Since AquaCrop-Hydro is partly physically based, the model requires calibration of just a few empirical model parameters using a transparent guided data-based approach. Calibration of AquaCrop crop parameters is only necessary when simulating new crops, or when soil fertility or salinity stress is considered. In those cases, identification of good parameter values is facilitated by the transparent stepwise calculation procedure of the model. Also, good indicative values for empirical soil parameters (surface runoff curve number and capillary rise parameters) can be obtained using AquaCrop's built-in default values which depend on soil type, land use and crop type. Furthermore, AquaCrop-Hydro’s routing model parameters, i.e. three recession constants, can be calibrated according to the stepwise procedure by \textcite{willems2014a}. This procedure is supported by subflow separation using the WETSPRO tool \parencite{willems2009}. Also, the baseflow-interflow separation equation can be calibrated from the filtered subflows.

Notwithstanding these low data and calibration requirements, this research confirmed that AquaCrop-Hydro can simulate crop production and water availability, as well as the effect of agricultural management and environmental changes, with reasonable accuracy. 

First, \autoref{ch:aquacrophydro} illustrated that AquaCrop-Hydro can simulate both crop production and water availability with acceptable accuracy. Crop yield in the Plankbeek catchment was estimated with a relative root-mean-square error (RRMSE) between 7 and 36.5\%, depending on the crop type. Also, discharge at the catchment outlet was simulated with satisfactory accuracy (model efficiency (EF) of 0.64) on a daily basis and high accuracy on a 10-day or monthly basis (EF of 0.82).

Second, \autoref{ch:fertility} and \ref{ch:weed} clearly illustrated the model's good balance between data requirements and accuracy for simulating the effect of agricultural management on the soil water balance, crop development and production at field scale. After calibrating the crop response to soil fertility stress based on easily obtainable inputs, the effect of soil fertility management can be simulated with a single input parameter, i.e. relative biomass production (\Brel). Also for weed management only two easily observable input parameters are required, i.e. the relative weed cover (\RC) and weed-induced increase of total canopy cover (\fweed). Nevertheless, crop production was simulated with a RRMSE between 4 and 26\% for five different crops (wheat, barley, maize, quinoa, tef) cultivated under various environmental conditions and different soil fertility, weed and water management treatments. In addition, AquaCrop performed very good for simulation of the soil water content in the root zone of these cropping systems, with RRMSE values between 4.5 and 13.5\%.

Finally, \autoref{ch:achtoep} illustrated that by straightforward adaptation of climate and management inputs, the effect of climate change and related management adaptations is simulated realistically. Although AquaCrop-Hydro's simulation results could not be directly validated to field observations, the projected agro-hydrological impact of climate change was in line with results of other simulation studies. Only the simulated crop responses to future weather conditions were not always as one would expect based on historical observations. This is caused by the above discussed limitations of AquaCrop to accurately simulate crop responses to water and (extreme) temperature stress. 

\subsection*{Criterion 4: The model is widely applicable, to various environmental conditions and cropping systems as well a agricultural catchments with different characteristics}
\textit{Assessment: Due to the wide applicability of both submodels, AquaCrop-Hydro fulfils this fourth criterion. Nevertheless there is potential to further expand the application domain.}

This research validated AquaCrop-Hydro for a single agricultural catchment, the Plankbeek catchment, in temperate Belgium (\autoref{ch:aquacrophydro}). Obviously, only validation for a wide range of agricultural catchments with varying characteristics can confirm that AquaCrop-Hydro is widely applicable. In particular, validation for drought-prone regions with rainfed cropping systems is important as those are the key areas for upgrading crop water productivity with improved agricultural management. Also, validation for data-scarce regions is needed to ensure that the model is applicable when few data are available, which is often the case in developing countries. Nevertheless, one can be confident that AquaCrop-Hydro is indeed widely applicable, because of the wide applicability of its submodels.

First, the list of publications composed by \textcite{vangaelen2016c} shows that AquaCrop has been applied to different cropping systems in more than 45 countries. These include developed countries such as USA \parencite{hsiao2009,heng2009}, Australia \parencite{zeleke2011} and Belgium \parencite{vanuytrecht2014}, as well as developing countries such as Ethiopia \parencite{abrha2012, tsegay2012}, Burkina Faso \parencite{wellens2013}, Iran \parencite{andarzian2011}, Bolivia \parencite{geerts2009a} and Nepal \parencite{shrestha2013}. The studied cropping systems cover a wide range of environmental conditions (arid to humid, tropical to temperate climatic conditions and various soil types), agronomic management practices (rainfed versus irrigated agriculture, various crop and field management practices) and crop types. AquaCrop has been applied for more than 30 different crops. These include widely cultivated crops such as barley, maize, wheat and rice \parencite{garciavila2009,heng2009,andarzian2011,abrha2012,shrestha2013}, but also under-utilized crops such as quinoa, tef and bambara groundnut \parencite{geerts2009,tsegay2012,karunaratne2011}. In addition, this research confirmed AquaCrop's wide applicability as the model was applied to 14 different crops, cultivated at nine locations with various agronomic and environmental conditions in seven different countries. 

Second, also conceptual models defined according to the VHM approach by \textcite{willems2014a} have been applied to catchments in different countries, including Belgium \parencite{vansteenbergen2012,willems2014},  Ecuador \parencite{moraserrano2013}, China \parencite{liu2011}, Uganda \parencite{nyekoogiramoi2010}, Kenya and Ethiopia \parencite{taye2011}. Catchments varied in climatic conditions, topography, soil types, land use and ecosystems. Catchment size ranged between 30 and \SI{15,000}{km^2}, but \autoref{ch:aquacrophydro} shows that the VHM approach is also applicable to very small catchments such as the Plankbeek catchment (\SI{4.5}{km^2}). Furthermore, the simple and flexible model structure of VHM-type conceptual models ensures that they are widely applicable to catchments with varying characteristics. Model components can be easily added or discarded depending on whether they are important in the study area and their properties identifiable from the available data.
  
Despite these promising indications of AquaCrop-Hydro's wide applicability for agricultural catchments, \autoref{ch:aquacrophydro} identified some opportunities to further expand the model application domain.

Although, in theory, the studied catchments can contain any crop, the use of AquaCrop-Hydro is strongly facilitated when the catchment crops are included in the AquaCrop database. This database contains default parameter sets that can be used with minimal additional calibration. Since the launch of AquaCrop in 2009, only tef and barley have been added to the original database of 12 crops. Currently, most important cereals (maize, wheat, barley, rice and sorghum) and some important tubers (potato and sugar beet) are included, but vegetable crops are under-represented (only tomato). Although validation of existing parameter sets for various environmental conditions is important, the research community should focus its efforts on developing new crop parameter sets. The Food and Agriculture Organization (FAO), being the model developer, should facilitate collaboration between different research groups in order to validate new parameter sets over a wide range of environmental conditions. Only then, a solid set of conservative parameters can be obtained for inclusion in the AquaCrop database. In addition, it would be practical if the AquaCrop database contained a parameter set for simulation of a bare soil with no canopy cover. Such a crop file would facilitate simulation of extensive crop rotations that contain fallow periods as presented in \autoref{ch:aquacrophydro} and \ref{ch:achtoep}.

Furthermore, it was found in \autoref{ch:aquacrophydro} that with the current model structure AquaCrop-Hydro should be used with caution when agricultural catchments contain a high proportion of:
\begin{itemize}
%\item \textbf{Impervious surface:} The above discussed poor estimation of surface runoff by AquaCrop-Hydro strongly distorts the soil water balance and water availability simulations for urban catchment where a large proportion of rainfall results in surface runoff. NOT TRUE, there is more runoff, but rainfall intensity is less importnat on asfalt than on soils. 
\item \textbf{Groundwater-dominated soils:} As discussed above, AquaCrop-Hydro's poor description of the two-directional interaction between the unsaturated and saturated zone, make model use infeasible and simulations of the soil water balance and hydrological processes inaccurate when a large area of the catchment has a shallow groundwater table.
\item \textbf{Winter crops:} As discussed above, the current AquaCrop simulation procedures lead to inaccurate simulation of development and production of winter crops. The poor estimation of evapotranspiration also affects the simulated soil water balance and water availability of catchments containing a large proportion of winter crops. 
\item \textbf{Grassland and fodder crops:} Although about 75\% of the world's agricultural area is covered by pasture and fodder crops \parencite{fao2008}, AquaCrop is not able to accurately simulate development and production of these crop types. First intents to use AquaCrop for simulation of alfalfa have been made by \textcite{kim2015}, and the obtained crop parameters were also applied in this research to simulate grassland and cover crops in the Plankbeek catchment (\autoref{ch:aquacrophydro}-\ref{ch:achtoep}). However, to improve simulation accuracy for crop canopy development and production, new calculation procedures need to be introduced. These procedures need to deal with the perennial character of these crop types, carry-over effects across different production seasons, as well grazing or cutting which reduces the crop canopy cover \parencite{holzworth2015,snow2014}. 
\item \textbf{Perrenial trees and woody crops:} AquaCrop was developed for simulation of herbaceous crops, but not woody crops or trees. Nevertheless, the model has been used to estimate evapotranspiration of olive tree \parencite{rallo2012} and jatropha \parencite{segerstedt2013}. The model also seems promising for estimation of leaf or wood biomass production, as shown by studies for tea \parencite{elbehri2015} and short rotation poplar plantations \parencite{horemans2016}. However, new procedures need to be implemented to deal with the perennial character of woody crops and trees, as well as the effect of biomass harvesting. Furthermore, it seems very unlikely that AquaCrop could obtain good results for fruit trees, as fruit formation is a highly complex process affected by tree pruning and weather conditions over several years which is not well understood. 
\end{itemize}

\section{Application of AquaCrop-Hydro}
AquaCrop-Hydro does not only meet the four criteria to a large extend, but has the additional benefit of being a very practical tool. It was demonstrated by the extensive scenario analyses in \autoref{ch:aquacropscen} and \autoref*{ch:achtoep} that AquaCrop-Hydro allows efficient analysis of several agricultural management scenarios for various environmental conditions. This efficiency stems from the short data processing and model execution time. 

Processing of the model input and calibration data is easy due to AquaCrop-Hydro's low data and calibration requirements, as discussed above. Furthermore, preparation of input data is facilitated by the user-friendly interface of the AquaCrop software. In addition, the AquaData tool or AquaCrop-GIS frameworks \parencite{lorite2013,thorp2013} can facilitate data processing for extensive simulation studies that require a large number of simulations for several locations or scenarios.

By contrast, including AquaCrop as a submodel increases internal data-processing time, because AquaCrop's protected source code impedes direct model linkage. The AquaCrop calculation procedures for simulation of the soil water balance and crop production at field scale could not be directly implemented in the AquaCrop-Hydro Matlab code \parencite{vangaelen2016d} that was developed in the framework of this research. In stead, the Matlab script invokes an AquaCrop simulation for each land unit and extracts the required data from the AquaCrop output files. These data are subsequently used for simulating the catchment soil water balance and water routing to the catchment outlet. Obviously, this internal data-processing significantly increases executions times of AquaCrop-Hydro. 

Despite the inefficient link between AquaCrop and the hydrological submodel, model execution times of AquaCrop-Hydro are still small. This is certainly so when AquaCrop simulations are done using the AquaCrop plugin software, which executes the model without a graphical user interface. \autoref{ch:aquacrophydro} demonstrated that a 15 year long AquaCrop-Hydro simulation for the Plankbeek catchment (31 land units) took less than 4 minutes using the Matlab code by \textcite{vangaelen2016d} and the AquaCrop plugin on a standard computer. This is significantly faster than the execution of the ArcNemo model for the same catchment (Van Opstal, personal communication) and typical SWAT executions for similar catchments \parencite{yalew2013}. 

Next to being a practical tool, AquaCrop-Hydro has upgraded evaluation of agricultural management from field to catchment scale. This cross-scale evaluation ensures that the trade-offs that arise when agricultural management affects water allocation within the catchment can be taken into account during the decision making process. More specifically, AquaCrop-Hydro reveals the trade-off between the on-site crop (water) productivity and off-site water availability impact of agricultural management. This is crucial, because \textcite{molden2010} report that increases in water productivity at farm level can increase basin water depletion. Conversely, \textcite{kijne2009} mention that increasing crop water productivity might lead to a reduction of water withdrawal for supplemental irrigation which positively affects water availability in the region. Also, the case-study of the Plankbeek catchment (\autoref{ch:achtoep}) demonstrated that adapted agricultural management increased crop yield and water productivity under future climatic conditions and at the same time increased water availability. 

AquaCrop-Hydro has not only up-scaled evaluation of agricultural management, but also down-scaled investigation of large scale environmental changes (e.g. climate change) to the scale of individual land units. \autoref{ch:achtoep} revealed how the complex effect of climate change on the plant-crop system also influences the overall effect of climate change on the agricultural catchment. Indeed, climate change has a direct impact on the catchment water balance because of altered weather conditions, but on top of that it also affects the water balance because of its effect on crop growth and evapotranspiration. 

Hence, AquaCrop-Hydro presents a first step towards evaluation of environmental changes and management actions across scales and research domains, an important prerequisite for integrated land and water resources management. Nevertheless, further integration of additional aspects related to agricultural management and its regional impact is required to conduct a fully integrated assessment of agricultural management practices. These include, for example, evaluation of the socio-economic feasibility of agricultural management practices and their effect on water quality (not just water quantity).

These additional aspects can only be simultaneously evaluated by fully integrated models, that consider all socio-economic and environmental aspects of land and water resources management within a catchment. However, upgrading AquaCrop-Hydro to such an integrated assessment model inevitably leads to loss of model simplicity, transparency and applicability. A better option would be to link AquaCrop-Hydro to additional models by means of a flexible modelling framework, in which model components can be added or removed according to the desired application. Unfortunately, the current generation of agro-hydrological models are rarely designed with attention for model reuse or linkage functionality. As such, their incorporation in large modelling frameworks is tedious. Clearly, the modelling community faces the challenge to develop models that (i) are open-source, (ii) are well documented, (iii) use standard data formats, and (iv) have a flexible model structure which consists of small independent modules that can easily be reused \parencite{laniak2013,holzworth2015,bergez2012}.

The release of an open-source AquaCrop code by FAO would be an important step in that direction. Especially, when this model-code is composed of small reusable model components, like the AquaCrop code currently being implemented in the BioMA (Biophysical Models Applications) platform \parencite{europeancommission2016}. Direct incorporation of such a open-source AquaCrop model code into AquaCrop-Hydro would reduce the model's execution time and facilitate further model development. In addition, it would promote linking AquaCrop-Hydro to other models in a fully integrated modelling framework. Only then, AquaCrop-Hydro can reach its full potential and contribute to support decisions regarding agricultural management, taking into account different spatial and temporal scales as well as multiple stakeholders with varying goals and incentives that share the precious natural resources.

\section{General conclusion}
AquaCrop-Hydro appears to be a parsimonious, widely applicable model that is able to evaluate the agro-hydrological impact of agricultural management from field to catchment scale. Hence, the developed model meets the targeted characteristics and consequently fills an important gap in the range of existing agro-hydrological models. With the current model structure, AquaCrop-Hydro can be used to evaluate the effect of agricultural management and environmental changes on crop development, crop production, crop water productivity, the soil water balance and water availability in agricultural catchments. Nevertheless, there is room to improve model accuracy, functionality and expand the application domain. Future model development could improve simulation of the interaction between the saturated and unsaturated zone for groundwater-dominated catchments. Also, revision of the surface runoff calculation procedures might enable model application for flood investigation. In addition, AquaCrop-Hydro would benefit from further developments of the AquaCrop submodel. Priority should be given to (i) improvement of simulation procedures describing crop responses to water and (extreme) temperature stress, (ii) addition of procedures to improve simulation of winter crops, grassland, forage and woody crops, and (iii) addition of new crops, especially vegetable crops, to the AquaCrop crop database. Finally, releasing the AquaCrop source code deems a prerequisite for further development of AquaCrop-Hydro, as well implementation of AquaCrop-Hydro in an extensive modelling framework to support fully integrative water and land resource management.


\cleardoublepage


% EXTRA
%
%DEFEFNITIE INTEGRATED WATER AND LAND RESOURCE MANAGEMENT zie kijne2009
%Nevertheless, decisions that influence water availability and allocation should be analysed in an even wider framework. Indeed, various stakeholders with different incentives all influence water availability and allocation. ZIE ARTIKEL MOLDEN 2010
%Voorbeel van fully integrated approach BERGEZ 2012
%voorbeel effect manaegemnt op WP en water  Wesseling 2006 (India case study)





%Another option would be for AquaCrop-Hydro to circumvent the AquaCrop surface runoff calculation module. Surface runoff could be calculated using an external model using sub-daily rainfall data. Next, the Aquacrop soil water balance can be simulated artificially setting surface runoff to zero and using input of effective rainfall, being the daily rainfall minus the externally simulated surface runoff. Obviously with this work-around option the link between the simulated soil water content and surface runoff is no longer considered. It is clear that a good solution can only be obtained when one would have direct access to the AquaCrop source code.

%For that reason, a linkage between AquaCrop and pest models such as XXX seem more appropriate for investigation of pest management. To date such a link has never been tested, but could be follow the example of X who linked X with a pest model. 

%Furthermore, more attention should be paid to correct calibration of the crop-specific conservative temperature thresholds. For example, the upper temperature for cotton development has been set at X and X  and X () by various authors, indicating lack of agreement on a accurate value for this parameter
% XXXX state that model validation is only a true validation when the model is applied to regions which vary sufficiently in environmental conditions. Not just varies crops and rainfall amounts should be covered, but also wide ranges of temperatures should be included in validation datasets. 

%Good results were obtained by \textcite{elbehri2015} for simulation of tea production with a test-version of AquaCrop that considered pruning of the tea-plants. Also, simulation of biomass production for bio-energy by short rotation poplar plantations appears to be promising \parencite{horemans2016}.


%Furthermore, caution should be paid to agricultural catchments that contain winter crops. The above discussed restrictions of AquaCrop to accurately simulate development and production of winter crops make that simulations for catchments containing a large proportion of these crops should be interpreted with care. The same is true for catchments that contain grassland, fodder crops, woody crops and trees.

%Although about 75\% the world agricultural area is covered by pasture and fodder crops \parencite{fao,2008}, AquaCrop (and consequently AquaCrop-Hydro) is not able to simulate these crop types. First intents to use AquaCrop for simulation of alfalfa have been made by \textcite{kim2015}, and the obtained crop parameters were also applied in this research to simulate grassland and cover crops in the Plankbeek catchment (\autoref{ch:aquacrophydro,ch:achtoep}). However, to improve accuracy of the simulation of canopy cover and production new calculation procedures need to be introduced to deal with the perennial character of these crop types, carry-over effects across different production seasons, as well grazing or cutting that disseminates the crop canopy cover \parencite{holzworth2015,snow2014}. 
 
%Besides grassland, also the inclusion of perennial trees and woody crops in AquaCrop would be useful, both for increasing the application domain of AquaCrop-Hydro to forested catchments as for simulation of bio-energy woody crops. AquaCrop was already used to estimate evapotranspiration by olive tree \parencite{rallo2012} and jatropha \parencite{segerstedt2013}, but not to estimate production. Simulation of fruit trees with AquaCrop appears difficult due to the perennial nature of the trees and the fruit production that is highly dependent on pruning and growth affected by weather conditions in previous years. Simulation of bushes and woody crops, for which one is only interested in biomass itself appears more attainable. Good results were obtained by \textcite{elbehri2015} for simulation of tea production. Furthermore, the application of AquaCrop for simulation of short rotation poplar plantations for biomass-energy production appears to be promising \parencite{horemans2016}. 

%Finally, AquaCrop-Hydro should not be applied to urban catchments that have a high proportion of impervious area. The source for this limitation lies in the poor estimation of surface runoff by AquaCrop-Hydro as discussed above. Inaccurate estimation of surface runoff will more strongly distort soil water balance and water availability simulations for urban catchment where a large proportion of rainfall results in surface runoff.

%Although there are clear opportunities to expand the application domain of AquaCrop-Hydro, the model appears to be applicable for most agricultural catchments. Caution should be paid to catchments containing winter crops, grassland, fodder crops and woody crops as further improvements are necessary to accurately simulate crop development, transpiration and production for these crop types. Moreover, application of AquaCrop-Hydro should be completely restrained when catchments contain a high proportion of impervious surface or forest, seen the above discussed limitations of AquaCrop to accurately simulate these land use types.

%As the time step of the AquaCrop submodel is fixed to one day, implementation of a sub-daily time step for surface runoff calculation would require adaptations to the model's source code. In addition, also the curve number equation would need to be replaced by an equation which is more suitable to small time steps (e.g Green and Ampt method). However, these adaptations are not likely to be implemented in AquaCrop as they are of limited benefit to the crop modelling community. Moreover, rainfall data at sub-daily time steps are not always available.

%Finally,  AquaCrop-Hydro's dynamic simulation of crop canopy cover and consequently evapotranspiration is of crucial importance for accurate simulation of the effect of agricultural management and environmental changes on catchment water availability. It was quantified for the Plankbeek catchment in \autoref{ch:achtoep} that neglecting crop responses to climate change, as often done in hydrological models, could lead to overestimation of evapotranspiration by about 10\% during future summer months. Because evapotranspiration is a very large component of the soil water budget, such overestimation could cause considerable errors to the simulated discharge at the catchment outlet. This error could be further quantified by comparing the discharge simulated by the dynamic AquaCrop-Hydro model under future climatic conditions with results obtained by a static conceptual hydrological model. Conceptual models defined according to the VHM approach by \textcite{willems2014a} are most appropriate for that purpose given their large similarity with AquaCrop-Hydro.